\chapter*{Resumen}

Desde su origen, \textit{VSCode4Teaching} ha venido siendo una extensión web para Visual Studio Code, un entorno de desarrollo integrado, que tiene como objetivo facilitar y potenciar la docencia de la programación informática para mejorar la educación en competencias digitales y en el ámbito de la informática, área en pleno crecimiento y promulgación a nivel global.

Para ello, \textit{VSCode4Teaching} permite a los profesores crear y gestionar cursos con ejercicios de programación que se basan en una plantilla inicial propuesta por ellos y, opcionalmente, una propuesta de solución al ejercicio. Los alumnos inscritos en los cursos completarán los ejercicios descargándose la plantilla y realizando sobre ella su propuesta propia de resolución, sincronizándola con el servidor para guardarla e informar en tiempo real a sus profesores de los avances realizados hasta finalizarla.

La memoria de este Trabajo Fin de Grado describe en profundidad el ciclo de desarrollo relativo al cuarto hito evolutivo del proyecto \textit{VSCode4Teaching}, en el que se implementa una aplicación web de navegador que incorpora los procesos de negocio que los usuarios ejecutaban en la extensión de Visual Studio Code, eliminando la obligatoriedad de uso de este entorno para alcanzar a un público objetivo mayor.

El \textit{software} del proyecto se organiza en una arquitectura cliente-servidor. El servidor, encargado del suministro, persistencia e interpretación de los datos, intercambia información con dos clientes: la extensión para Visual Studio Code y la aplicación web de navegador, que disponen en consecuencia las interfaces gráficas necesarias para la interacción con la aplicación.

El proyecto \textit{VSCode4Teaching} es \textit{software} libre divulgado bajo licencia Apache 2.0 a través de un repositorio público en GitHub\footnote{Repositorio: \href{https://github.com/codeurjc-students/2019-VSCode4Teaching}{https://github.com/codeurjc-students/2019-VSCode4Teaching}.} que contiene, además, documentación sobre el proyecto para favorecer la libre ejecución, utilización y adaptación del proyecto a toda la comunidad de desarrolladores.

\vspace{2pt}

\noindent \textbf{Palabras clave}: educación, informática, programación, desarrollo de aplicaciones web, evolución del \textit{software}, mantenimiento \textit{software}.


\chapter*{Abstract}

Since its inception, \textit{VSCode4Teaching} has been a web extension for Visual Studio Code, an integrated development environment, aimed at facilitating and enhancing the teaching of computer programming to improve education in digital skills and in the field of computer science, an area in full growth and promulgation at a global level.

To this end, \textit{VSCode4Teaching} allows teachers to create and manage courses with programming exercises based on an initial template proposed by them and, optionally, a suggested solution to the exercise. Students enrolled in the courses will complete the exercises by downloading the template and providing their own developed solutions, synchronizing it with the server to save their work and inform their teachers in real-time of their progress until completion.

This Bachelor's Thesis report provides an in-depth description of the development cycle related to the fourth milestone of the \textit{VSCode4Teaching} project, which involves implementing a web application that incorporates the business processes that users used to execute in the Visual Studio Code extension, thereby eliminating the mandatory use of this environment to reach a broader target audience.

The project's \textit{software} is organized in a client-server architecture. The server, in charge of data supply, persistence and interpretation, exchanges information with two clients: the Visual Studio Code extension and the web browser application, which consequently provide the necessary graphical interfaces for interaction with the application.

The \textit{VSCode4Teaching} project is free \textit{software} released under the Apache 2.0 license through a public repository on GitHub\footnote{Repository: \href{https://github.com/codeurjc-students/2019-VSCode4Teaching}{https://github.com/codeurjc-students/2019-VSCode4Teaching}.}, which also contains documentation about the project to encourage the free execution, use, and adaptation of the project by the developer community.

\vspace{2pt}

\noindent \textbf{Keywords}: education, computer science, programming, web applications development, \textit{software} evolution, \textit{software} maintenance.
