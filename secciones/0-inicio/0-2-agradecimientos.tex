\chapter*{Agradecimientos}

A mis educadores vitales, mi familia, los que me han enseñado y allanado el camino, los que han dedicado su esfuerzo a formarme y a darme las facilidades que me han permitido crecer como persona. A mis abuelos, que se fueron muy pronto, gracias por dedicar vuestros últimos años a acompañarme durante mis primeros pasos; y en especial a Santos, el más longevo de ellos, el que vio con orgullo a su único nieto siendo el primer universitario de la familia. Gracias a ellos también por darme a mis padres, sempiternos baluartes, a quienes, a pesar de todos los contratiempos, les debo y les deberé siempre la vida tan fácil que me han regalado incondicionalmente.

A mis educadores intelectuales, a los profesores de los que he aprendido lo que sé durante los últimos veinte años. Gracias a los que me enseñasteis a razonar, analizar y aprender: a Antonio, a Teresa, a Silvia, a María Jesús y a tantos y tantos del Marillac. También a los de la Universidad, en especial a los que vivís con pasión vuestra gran profesión: a Sofía, a Mayte, a Ana y, en especial, a Mica.

A mis educadores sociales, a los que me habéis enseñado el valor de una buena amistad: a mis bachilleres, Miguel, Flor y Ana, que me habéis demostrado que, por mucho que diverjan los caminos, hay amistades que persisten; y a mis niños, José Luis, Andrea, Ismael, Debi y Flavia, gracias por todo lo que hemos disfrutado juntos dentro y, especialmente, fuera de la Universidad. Ojalá siga siendo así.

A mis educadores laborales, con los que he ``estrenado'' las ingenierías que todavía no tenía: al departamento de Aplicaciones Corporativas. Aunque este TFG no esté relacionado con nuestro cometido allí, de vosotros he aprendido cómo se aterriza nuestra fantástica profesión en el mundo real. Gracias a Juancar por nuestros debates sobre diseño y a todos los demás compañeros, en especial a mis coetáneos, a Gus, Julián y los Diegos, de los que he aprendido mucho. También a Santi, que hace que el trabajo sea mucho más llevadero.

Aun a riesgo de sonar petulante, termino agradeciéndome a mí las horas y el esfuerzo dedicados a cumplir el sueño que tengo desde hace quince años.

\begin{center}
    \textbf{Gracias}.
\end{center}