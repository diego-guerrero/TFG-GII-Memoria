\subsection{Divulgación, despliegue y distribución}
\label{subsec:tecDivulgDistribDeploy}

Si bien se confiere al diseño e implementación del \textit{software} un papel preponderante dentro del ciclo de desarrollo, existen otras tareas fundamentales que las circundan y que cobran una especial importancia al finalizar cada iteración del proceso \textit{software}: la divulgación del código fuente, especialmente reseñable en proyectos de código abierto como \textit{VSCode4Teaching}; la distribución de las nuevas versiones publicadas al conjunto de usuarios finales y el despliegue de las actualizaciones son parte esencial del proceso \textit{software}. Se introducen a continuación las principales tecnologías empleadas para la consecución de estas tareas, si bien a estas se suman algunas herramientas descritas en la \referenciaSeccion{sec:herramientas}, tales como GitHub Actions, el sistema empleado para la integración y entrega continuos.

\subsubsection{Divulgación del código: GitHub}
\label{subsec:tecGitHub}
GitHub es la ``plataforma empleada para almacenar, compartir y trabajar junto con otros usuarios para escribir código'' \cite{GitHub} más extendida en la comunidad de desarrolladores de \textit{software}, ya que alberga más de 420 millones de repositorios de código fuente creados por más de 100 millones de desarrolladores \cite{GitHubCifras}.

Si bien GitHub comenzó siendo un servidor remoto donde alojar repositorios de código fuente basados en el uso de \textit{git} como sistema de control de versiones, ha venido incorporando numerosas herramientas asociadas que lo consolidan hoy como un ecosistema para no solo almacenar y divulgar proyectos \textit{software}, sino también para trabajar sobre ellos en entornos remotos (GitHub Codespaces), facilitar la colaboración de los desarrolladores de los proyectos y gestionar la seguridad de los proyectos mediante mecanismos automatizados para la detección de vulnerabilidades, entre otras herramientas de utilidad para la gestión de proyectos.

GitHub viene siendo empleado como sistema de almacenamiento remoto del proyecto \textit{VSCode4Teaching} desde su inicio. El presente hito evolutivo, además, propone un uso aún mayor de las ventajas que aporta GitHub mediante la utilización de GitHub Actions que, desarrollado en la \referenciaSeccion{subsec:herCiCd}, es la herramienta que se emplea para el nuevo sistema de integración continua del proyecto.

La \referenciaSeccion{subsec:distribFuente} incluye más información acerca de su utilización como punto de divulgación del código fuente como código abierto y \textit{software} libre.

\subsubsection{Despliegue del servidor y la aplicación web: Docker}
\label{subsec:tecDocker}
Docker es ``una plataforma diseñada para ayudar a los desarrolladores a construir, compartir y ejecutar aplicaciones modernas'' \cite{Docker}. Alrededor de Docker existe un amplio abanico de tecnologías disponibles, entre las que cabe destacar su motor de ejecución, \textit{Docker Engine}, que está disponible como código abierto bajo licencia Apache 2.0 y que dispone de un servidor capaz de ejecutar imágenes en contenedores y de una interfaz por línea de comandos para operar con él \cite{DockerEngine}.

En este ámbito, se conoce como contenedor a una ``unidad estandarizada para el desarrollo, entrega y despliegue de paquetes \textit{software} que empaqueta el código y todas sus dependencias para ejecutar aplicaciones rápida y fiablemente'' \cite{DockerContainers}. La principal ventaja derivada del uso de contenedores como entorno de ejecución de aplicaciones es que la posibilidad de dotar a cada contenedor de un entorno aislado que cumpla los requerimientos \textit{software} necesarios para cada aplicación con independencia de la plataforma sobre la que se ejecute el motor de ejecución de Docker, proporcionando ventajas semejantes a la virtualización sin necesidades \textit{hardware} tan altas como las que requiere este formato. La independencia de la plataforma viene dada por el uso de imágenes, que son las especificaciones sobre las condiciones \textit{software} en las que debe ejecutarse un determinado código fuente al introducirse en un contenedor; esto es, las imágenes se especifican y generan para dar lugar a contenedores que las ejecutan. La especificación de imágenes Docker se realiza a través de los \textit{Dockerfile}, que son los ``ficheros de texto que contienen las instrucciones para generar una imagen a partir de un código fuente [\dots] empleando una sintaxis específicamente definida'' \cite{Dockerfile}.

Del ecosistema de tecnologías alrededor de Docker, también se utiliza Docker Compose en este proyecto. Docker Compose es una ``herramienta que permite definir y ejecutar aplicaciones basadas en múltiples contenedores'' \cite{DockerCompose}, pudiendo así redactar instrucciones, declaraciones y especificaciones mediante una determinada sintaxis para orquestar el funcionamiento de distintos contenedores en un entorno compartido, tal como se requiere para ejecutar \textit{VSCode4Teaching} junto con una instancia de un sistema gestor de bases de datos empleado para la persistencia.

El detalle acerca del requisito \referenciaConTT{subsec:rn4}{RN-4} incluye información acerca de la utilización del ecosistema Docker durante este nuevo hito evolutivo de \textit{VSCode4Teaching}.

\subsubsection{Distribución: Docker Hub y Visual Studio Code Marketplace}
\label{subsec:tecDistrib}
Además de las tecnologías desarrolladas en la sección anterior, el proyecto \textit{VSCode4Teaching} también hace uso de Docker Hub, que es el ``repositorio de imágenes más grande del mundo, incluyendo imágenes de la comunidad de desarrolladores y de proyectos de código abierto'' \cite{DockerHub}. La imagen Docker generada en el proyecto \textit{VSCode4Teaching}, que permite ejecutar el servidor y la aplicación web, se publica en Docker Hub.

Por otro lado, la extensión para Visual Studio Code queda divulgada a través del Visual Studio Code Marketplace, que es un repositorio público para la publicación de extensiones para este IDE disponible para la descarga e instalación de extensiones desde dentro del propio entorno, permitiendo así a los usuarios pueden buscar extensiones publicadas y descargarlas en su instancia local del editor de forma gratuita, rápida y sencilla \cite{VSCodeMarketplace}.

La \referenciaSeccion{subsec:distribArtefactos} incluye información detallada sobre cómo se hace uso de estos repositorios en el proyecto \textit{VSCode4Teaching}.
