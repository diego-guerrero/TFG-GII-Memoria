\subsection{Extensión para Visual Studio Code}
\label{subsec:tecExtension}
Se introducen en esta sección las tecnologías empleadas para la construcción de la extensión de \textit{VSCode4Teaching} para el entorno de desarrollo integrado Visual Studio Code (véase la \referenciaSeccion{subsec:herIDEs} para más información sobre esta herramienta). La extensión hace uso de \textbf{Node} como plataforma subyacente, está implementada mediante el lenguaje \textbf{TypeScript}, se basa en la \textbf{Visual Studio Code Extension API} como biblioteca fundamental y utiliza \textbf{Jest} para la codificación y ejecución de las pruebas automáticas.

\paragraph{Node y TypeScript}\mbox{} \\
Tal como se desarrolla en la \referenciaSeccion{subsec:tecAppWeb}, este cuarto hito evolutivo del proyecto introduce una nueva aplicación web del lado cliente basada en Angular. Algunas de las tecnologías fundamentales empleadas para la implementación de este nuevo componente ya se venían utilizando con la extensión: en ambos casos se hace uso de Node como plataforma subyacente y de TypeScript como lenguaje de programación. Esta compartición reduce la curva de aprendizaje de las tecnologías empleadas para el proyecto, facilitando posteriores tareas de mantenimiento.

\paragraph{Visual Studio Code Extension API}\mbox{} \\
Visual Studio Code es un entorno de desarrollo ``extensible y personalizable'' \cite{VSCode}, siendo esta una de las características más destacadas por sus creadores. Esta versatilidad de la herramienta queda materializada en su integración con el \textit{Marketplace}, punto para la descarga y divulgación de extensiones (véase la \referenciaSeccion{subsec:tecDistrib} sobre este método de publicación). En su implementación, las extensiones hacen uso de la Visual Studio Code Extension API, interfaz integrada en el IDE para la interacción con sus elementos nativos propios, permitiendo numerosas características entre las que cabe destacar: implementación de nuevos comandos, modificación de la estética del editor de código, visualización de notificaciones y elementos gráficos para la interacción del usuario (tales como campos de texto o listas de selección múltiple, entre otros), mostración del estado o progreso de la extensión en la barra de actividad, introducción de vistas web personalizadas en pestañas nuevas, adición de funcionalidad para la depuración, acceso al control de las ventanas y pestañas abiertas, generación de nuevos ítems en la barra lateral y modificación de las carpetas que conforman el área de trabajo activa.

\paragraph{Jest}\mbox{} \\
Jest es una herramienta para la implementación y posterior ejecución de \textit{tests} en aplicaciones basadas en JavaScript que ``pone el foco en la simplicidad''. Este proyecto, \textit{software} libre divulgado bajo licencia MIT, fue originalmente creado por Facebook y es mantenido en la actualidad por la \textit{Open JS Foundation} \cite{Jest}.

De una forma sencilla y sin necesidad de configuración adicional, Jest incorpora mecanismos para la generación de baterías de pruebas automáticas dando un soporte sencillo e intuitivo a las aserciones y los dobles, disponiendo de una utilidad para, mediante línea de comandos, ejecutar las pruebas y obtener informes sobre la cobertura del código y el detalle pormenorizado de las excepciones producidas en las pruebas erróneas.
