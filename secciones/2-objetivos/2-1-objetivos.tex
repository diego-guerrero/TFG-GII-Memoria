\chapter{Objetivos}
\label{cap:objetivos}

El \referenciaCapitulo{cap:introduccion} ha introducido pormenorizadamente la contextualización sociocultural que constata la necesidad del proyecto \textit{VSCode4Teaching}, una cronología de las distintas evoluciones que lo han ido conformando y la necesidad de generar una nueva interfaz para que los usuarios puedan utilizar esta herramienta sin necesidad de hacer uso de Visual Studio Code.

El presente Trabajo Fin de Grado busca alcanzar el cuarto hito en la evolución del proyecto, poniendo el foco en la necesidad anteriormente descrita con el fin de alcanzar a la mayor cantidad de potenciales usuarios posible, facilitando el acceso y la utilización de la herramienta. Esta necesidad es la piedra angular de este Trabajo Fin de Grado, que fija como su objetivo principal alcanzar al mayor público objetivo posible.

Traducido al plano técnico, este objetivo se va a materializar a través de la \textbf{implementación de un componente nuevo} en \textit{VSCode4Teaching}: una aplicación web que actúe como cliente (esto es, un \textit{frontend}), recogiendo toda la funcionalidad posible de la extensión y haciéndola independiente de Visual Studio Code, desapareciendo así el único requisito \textit{software} que se imponía a los usuarios. De este modo, la nueva aplicación web permitirá a sus usuarios ejecutar todos aquellos procesos de negocio que venían pudiendo realizar en la extensión para Visual Studio Code y que no dependiesen específicamente del citado entorno de desarrollo. Este matiz conduce a la exclusión de características como, por ejemplo, la capacidad para introducir comentarios textuales asociados a las líneas de código de los ficheros, característica intrínsecamente vinculada al entorno de desarrollo.

El proyecto ya dispone de una pequeña aplicación web implementada mediante Angular que se emplea como componente auxiliar para la ejecución de funcionalidades como el registro en dos partes de nuevos docentes mediante invitación y la visualización de una página de ayuda personalizada para alumnos cuando reciben una invitación a un curso por parte de un profesor. Como el alcance pretendido es más amplio y se dirige a la obtención de un componente renovado y completo, el objetivo principal de este Trabajo Fin de Grado en el plano técnico es el de generar una nueva aplicación web desde cero que permita alcanzar el objetivo de dominio, reemplazando el anterior \textit{frontend} existente.

Por todo lo anterior, cabe concluir que esta nueva evolución del proyecto \textit{VSCode4Teaching} busca acoplar al proyecto una aplicación web de lado cliente que permita a estudiantes y profesores operar con sus cursos y ejercicios, pudiendo ejecutar todos los procesos de negocio posibles de los previamente existentes. Este hecho conduce a que objetivos de dominio de este Trabajo Fin de Grado se fundamenten en los objetivos de negocio establecidos en las tres iteraciones previas, entre los que cabe destacar los siguientes seis objetivos:
\begin{enumerate}
    \item \underline{Autenticación y visualización personalizada de cursos y ejercicios}: permitir a los usuarios autenticarse, de modo que cada usuario podrá visualizar sus cursos inscritos o impartidos y los ejercicios que los compongan.
    \item \underline{Impartición de cursos y ejercicios por docentes}: brindar a los docentes las herramientas para gestionar sus cursos y añadir en ellos ejercicios a partir de plantillas iniciales y, opcionalmente, incorporándoles propuestas de solución que estarán disponibles al estudiantado cuando el docente lo desee.
    \item \underline{Seguimiento activo de ejercicios para docentes}: dotar a los docentes de las herramientas necesarias para seguir el progreso de los estudiantes al realizar los ejercicios de sus cursos, obteniendo información actualizada en tiempo real y pudiendo descargar sus propuestas de resolución.
    \item \underline{Matriculación del estudiantado en cursos}: permitir a los docentes gestionar los estudiantes que están matriculados en sus cursos y, además, compartirlos para permitir a los estudiantes automatricularse en ellos.
    \item \underline{Realización de ejercicios en cursos matriculados por alumnos}: otorgar a los estudiantes la capacidad para descargar los ficheros asociados a los ejercicios de los cursos de los que forman parte para realizarlos, sincronizando las modificaciones que vayan realizando según se produzcan.
    \item \underline{Calidad del proyecto y de su \textit{software}}: realizar actuaciones para mantener y mejorar el propio proyecto \textit{software}, migrando y ampliando el sistema de integración continua y, además, mejorando y automatizando la generación de los artefactos compilados de la aplicación para adecuar su despliegue y distribución al nuevo componente introducido.
\end{enumerate}