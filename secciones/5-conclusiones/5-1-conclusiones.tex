\chapter{Conclusiones y trabajos futuros}
\label{cap:conclusiones}

Esta última sección concluye el Trabajo Fin de Grado realizado, para lo que se analiza el cumplimiento de los objetivos estipulados ---\referenciaSeccion{subsec:cumplimientoObjetivos}---, se introducen varios posibles trabajos futuros de interés para el proyecto \textit{VSCode4Teaching} ---\referenciaSeccion{subsec:trabajosFuturos}--- e incluyendo, para finalizar, una opinión personal final acerca del Trabajo Fin de Grado ---\referenciaSeccion{subsec:aprendizajesPersonales}---.

\section{Cumplimiento de los objetivos estipulados}
\label{subsec:cumplimientoObjetivos}
El actual Trabajo Fin de Grado supone un cuarto hito evolutivo en el proyecto \textit{VSCode4Teaching}, basado en los objetivos iniciales estipulados recogidos en el \referenciaSeccion{cap:objetivos} sobre los que, a su vez, se asientan los requisitos especificados en la \referenciaSeccion{sec:requisitos} y cuyo diseño e implementación queda recogido en las subsiguientes secciones. Se confirma el cumplimiento de la totalidad de los objetivos establecidos, tal como se detalla a continuación:

\begin{itemize}
    \item El objetivo primero establece la necesidad de que los usuarios registrados puedan autenticarse (\referenciaConTT{subsec:rf1}{RF-1.1}) y, cuando hayan iniciado sesión de forma exitosa, puedan ver sus cursos matriculados, en el caso de los estudiantes; o impartidos, en el caso de los docentes, quienes también disponen de una visualización del detalle de un curso que refleja los ejercicios que contiene, aportando información básica adicional y acceso a las opciones de gestión del curso (\referenciaConTT{subsec:rf1}{RF-1.2}).
    
    Este objetivo también viene ratificado por la implementación del \referenciaConTT{subsec:rf9}{RF-9}, por el que los estudiantes disponen de una pantalla en la que, una vez elegido un directorio de su sistema local de ficheros, pueden visualizar con detalle el estado de ejecución de los ejercicios de sus cursos.
    Además, cabe reseñar la utilidad del requisito \referenciaConTT{subsec:rn2}{RN-2} en este objetivo, ya que proporcionar un aspecto visual coherente con la extensión para Visual Studio Code facilita a los usuarios previamente registrados la utilización y comprensión de la nueva aplicación web.
    \item El segundo objetivo especifica la necesidad de los docentes de disponer de la capacidad para añadir y gestionar los ejercicios de sus cursos. Este objetivo viene implementado por dos requisitos funcionales: \referenciaConTT{subsec:rf2}{RF-2}, por el que los docentes pueden añadir nuevos ejercicios en sus cursos de forma individual o en lotes de ejercicios con o sin propuesta de solución; y el \referenciaConTT{subsec:rf5}{RF-5}, que establece la necesidad de configurar la visibilidad de la solución propuesta por el profesor y la capacidad para realizar nuevas ediciones tras ser descargada por los estudiantes.

    Este objetivo, además, viene complementado por el \referenciaConTT{subsec:rn1}{RN-1}, por el que se muestra un aviso por pantalla en caso de utilizar un navegador no compatible con la interfaz para la interacción con los sistemas de ficheros locales, empleada para la creación de nuevos ejercicios.

    \item El objetivo tercero marca la necesidad de que los docentes dispongan de herramientas para el seguimiento en tiempo real del progreso de los estudiantes inscritos en sus cursos impartidos. Su intención queda materializada en la implementación de los requisitos \referenciaConTT{subsec:rn3}{RN-3}, por el que los docentes disponen de un \textit{dashboard} para cada ejercicio de sus cursos actualizado en tiempo real que contiene numerosas métricas acerca del progreso de los alumnos; y \referenciaConTT{subsec:rn4}{RN-4}, que añade a la interfaz anteriormente citada la posiblidad de elegir un directorio de sistema local de ficheros para descargar en él bajo demanda los ficheros que integran las propuestas de resolución de ejercicios de los estudiantes. Este último requisito conduce a la mención del valor aportado del requisito \referenciaConTT{subsec:rn1}{RN-1}, ya que la característica para la descarga no está disponible en navegadores no compatibles con la \textit{File System Access API}.

    \item El cuarto objetivo complementa las necesidades de los docentes en \textit{VSCode4Teaching} especificando la necesidad para la gestión de la matriculación de estudiantes en los cursos que imparten. Con este fin, el requisito \referenciaConTT{subsec:rf6}{RF-6} introduce la capacidad para la gestión de los estudiantes inscritos en los cursos a través de una interfaz visual que permite revocar inscripciones o matricular a nuevos usuarios. Adicionalmente, los docentes disponen de la capacidad para compartir un código único de inscripción asociado a sus cursos (\referenciaConTT{subsec:rf7}{RF-7}) que puede ser empleado por los estudiantes para su automatriculación (\referenciaConTT{subsec:rf8}{RF-8}).
    
    \item El objetivo quinto pone el foco en los estudiantes, quienes deben poder visualizar los ejercicios que componen sus cursos y realizarlos, sincronizando las modificaciones que realicen en sus propuestas de resolución. Este objetivo queda implementado a través de cuatro requisitos funcionales encadenados: la incorporación de la interfaz de usuario que permite a los estudiantes escoger un directorio del sistema de ficheros de su computador y obtener el detalle del progreso en la ejecución de los ejercicios del curso (\referenciaConTT{subsec:rf9}{RF-9}), poder descargar las plantillas iniciales de los ejercicios recién comenzados o el último punto de progreso sincronizado en el servidor en el directorio escogido (\referenciaConTT{subsec:rf10}{RF-10}), sincronizar automáticamente con el servidor el progreso de los ejercicios cada vez que se crea, modifica o elimina un fichero en el sistema local del estudiante (\referenciaConTT{subsec:rf11}{RF-11}) y otorgar a los estudiantes la capacidad para marcar sus propuestas de resolución como finalizadas, impidiendo nuevas ediciones (\referenciaConTT{subsec:rf12}{RF-12}). Cabe mencionar nuevamente la utilidad del requisito \referenciaConTT{subsec:rn1}{RN-1} y el valor que aporta a la ejecución exitosa del presente objetivo.
    
    \item El sexto objetivo hace énfasis en la calidad del proyecto, procurando la ejecución de actuaciones para la mejora de sus atributos de calidad. Entre las actuaciones realizadas, destaca la adición de una capa de seguridad extra en el mecanismo de autenticación (\referenciaConTT{subsec:rn3}{RN-3}), la adecuación de la generación de artefactos a la nueva arquitectura adoptada en el proyecto (\referenciaConTT{subsec:rn4}{RN-4}) y la mejora y automatización integral de los procesos de integración, entrega y despliegue continuos (\referenciaConTT{subsec:rn5}{RN-5}).
\end{itemize}

\section{Trabajos futuros: hoja de ruta del proyecto \textit{VSCode4Teaching}}
\label{subsec:trabajosFuturos}
\textit{VSCode4Teaching} es un proyecto que, tal como se refleja en la \referenciaSeccion{sec:cronologiaProyecto}, es el resultado de, junto con el presente, cuatro Trabajos Fin de Grado que marcan sobre él cuatro hitos evolutivos.

El actual Trabajo Fin de Grado no ha completado la migración de la extensión al formato de aplicación \textit{web}, aunque la funcionalidad principal de la aplicación ya puede ser ejecutada a través de esta nueva interfaz de usuario. La actual situación del proyecto permite plantear una extensa batería de posibles trabajos futuros de gran interés y valor para el usuario.

\noindent Para la mejora de la aplicación \textit{web} de \textit{VSCode4Teaching}, cabe considerar:
\begin{enumerate}
    \item Finalizar algunas de las características más complejas, tales como la gestión pormenorizada de cursos y ejercicios, así como perfeccionar el funcionamiento de las características existentes, realizando sobre la aplicación nuevos y constantes trabajos de mantenimiento correctivo y perfectivo.
    \item Incorporar la posibilidad de visualizar los contenidos de los ejercicios y de modificarlos a través de un editor de código integrado ligero que quede embebido en la aplicación \textit{web}. Aunque puede ser muy complejo dotar a la herramienta de la capacidad para ejecutar los proyectos de los alumnos, es interesante disponer un editor mínimamente interactivo que permita visualizar los contenidos de los ejercicios o hacer ediciones rápidas sobre ellos sin necesidad de utilizar un directorio en el sistema local de ficheros, permitiendo así, además, ver o editar ejercicios en navegadores no compatibles con la \textit{File System Access API}. Por ejemplo, el editor que emplea Visual Studio Code, llamado \textit{Monaco} \cite{MonacoEditor}, está disponible como \textit{software} libre bajo licencia MIT y puede ser integrado en cualquier aplicación \textit{web}.
    \item Implementar una amplia batería de pruebas automáticas de diversa índole: pruebas unitarias asociadas a la lógica de negocio, pruebas de integración para garantizar la correcta comunicación con el servidor y pruebas de sistema o \textit{end to end} (E2E) para ratificar el correcto funcionamiento de los elementos dispuestos en la interfaz de usuario y la adecuación de la implementación de los distintos procesos de negocio disponibles para los usuarios.
\end{enumerate}

Por otro lado, en el actual punto de evolución de \textit{VSCode4Teaching} como proyecto \textit{software}, cabe analizar si es pertinente continuar manteniendo la extensión para Visual Studio Code en paralelo a la aplicación \textit{web} y disponer de dos clientes plenamente funcionales disponibles para los usuarios. Esta decisión debe dirimirse tomando en consideración varios factores, entre los que cabe destacar la dificultad y el coste temporal que conlleva el mantenimiento paralelo de dos clientes con la misma funcionalidad y finalidad.

En particular, si se decidiese continuar manteniendo ambos clientes de forma paralela, podría ser interesante plantear una reestructuración de su código para abstraer la funcionalidad común a ambos componentes a una biblioteca propia del proyecto. Como ambos clientes se basan en la plataforma Node y en la gestión de paquetes realizada a través de NPM, tal como se detalla en la \referenciaSeccion{subsec:tecAppWeb} y la \referenciaSeccion{subsec:tecExtension}, puede resultar conveniente generar un paquete que abstraiga toda la lógica de negocio y la interacción con el servidor, que es compartida por ambas aplicaciones y que actualmente se realiza utilizando bibliotecas y mecanismos diferentes, logrando así eliminar gran parte de la duplicidad de código entre ambos clientes y, por tanto, facilitando el esfuerzo de mantenimiento, de modo que cada uno deberá contener únicamente el código necesario para la mediación entre la interfaz de usuario ---necesariamente específica de cada uno--- y la lógica de negocio, disponible a través de servicios comunes.

Otro trabajo futuro conveniente a realizar en \textit{VSCode4Teaching} es implementar una mejora que permita extender el uso del \textit{Web Socket} existente para ampliar su funcionalidad y dotar a los clientes de una interacción integral en tiempo real. Para ello, se puede establecer una política de uso del \textit{Web Socket} que permita identificar qué eventos son de interés para cada usuario según sus cursos impartidos o matriculados y, en consecuencia, notificar la ocurrencia de eventos para mantener actualizadas de inmediato todas las interfaces gráficas que los usuarios conectados estén empleando.

Adicionalmente, es susceptible de mejora el sistema de transmisión de los ficheros entre servidor y clientes. Actualmente, la extensión para Visual Studio Code genera un fichero comprimido con la totalidad del contenido de los ejercicios cada vez que se persiste una sola modificación de un archivo, lo que hace que esta operación sea costosa en tiempo y en recursos. Se puede unificar la forma en que se envían las modificaciones parciales realizadas por los estudiantes y remitirlas igual que en el caso de la aplicación \textit{web}, que ciñe la transmisión al envío de cada fichero creado, modificado o eliminado. Esta ``atomización'' de las tareas permitirá potenciar la eficiencia de la comunicación entre cliente y servidor, pudiendo preservar el formato de transmisión de archivos comprimidos cuando se requiera el envío de una ingente cantidad de ficheros ---por ejemplo, al subir la plantilla de los ejercicios o cuando un docente descarga por primera vez los ficheros de las propuestas de los alumnos---.


Aparte de estos objetivos, la hoja de ruta de \textit{VSCode4Teaching} recoge propuestas de nueva funcionalidad para incorporar en el proyecto, tales como:
\begin{enumerate}
    \item Dotar a los ejercicios de mayor funcionalidad, permitiendo a los docentes características como: escoger su visibilidad hacia el estudiantado, configurar un único ejercicio como ``activo'', permitir escoger el orden en que se muestran, determinar la fecha máxima de finalización o estipular un periodo de tiempo máximo permitido para su realización.
    \item Añadir soporte a las calificaciones, de modo que los docentes puedan puntuar las propuestas de resolución de ejercicios elaboradas por los estudiantes y aportarles realimentación, permitiendo al alumnado revisar las calificaciones otorgadas. Además, se considera la posibilidad de introducir estrategias para la detección de plagio que permitan determinar de forma automática la calificación de los ejercicios por comparación con la propuesta de solución del docente o con las demás propuestas del estudiantado.
    \item Introducir un nuevo rol de administración que disponga de capacidades específicas para la gestión completa de cursos, ejercicios y usuarios.
    \item Explorar la integración del proyecto con entornos LMS\footnote{LMS. Siglas de ``sistema de gestión de aprendizaje'' (del inglés \textit{Learning Management System}).} y, en particular, con el Aula Virtual de la Universidad Rey Juan Carlos, basado en Moodle \cite{Moodle}, mediante el uso de un interfaz LTI\footnote{LTI. Siglas de ``interoperabilidad entre herramientas de aprendizaje'' (del inglés \textit{Learning Tools Interoperability}).} \cite{LTI_Spec}.
\end{enumerate}

\section{Aprendizajes personales}
\label{subsec:aprendizajesPersonales}
Permítaseme redactar en primera persona la sección final de la memoria de mi segundo Trabajo Fin de Grado, en la que busco plasmar las conclusiones personales más destacadas de esta experiencia.

Tal como ya dije al concluir mi primera memoria, los Trabajos Fin de Grado que he escogido son una excelente forma de trasladar al plano práctico tantos conocimientos teóricos recibidos durante las asignaturas del itinerario formativo de mi doble grado: después de estar cuatro años recibiendo una formación teórica amplia y de calidad sobre la ingeniería informática y la ingeniería del software, trabajar en \textit{VSCode4Teaching} me ha permitido trasladar a la práctica 

Mis Trabajos Fin de Grado me han permitido acercarme a la rama de mis ingenierías que más disfruto, que es el desarrollo de aplicaciones web y todo lo que conlleva: la toma de decisiones de diseño y arquitectura, la utilización de un enorme abanico de tecnologías disponibles, el necesario continuo aprendizaje de los estándares más elementales, que están en constante evolución; la filosofía DevOps y la automatización de los procesos del software, la implementación de nuevos requisitos en servidor y cliente\dots un sinfín de tareas que envuelven la rama a la que ya me dedico profesionalmente y de la que quiero seguir aprendiendo más y más.

Toca ahora seguir aprendiendo, ya que me dedico con auténtica vocación a una disciplina muy exigente, en plena expansión, que cada vez dispone de más y más opciones para que los desarrolladores generemos mejores herramientas y, al fin y al cabo, para que cada vez ayudemos mejor a los usuarios generando \textit{software} que aporte más valor y sea de mejor calidad.

Si los planes salen adelante, el punto que termina este párrafo no será mi punto final en el proyecto \textit{VSCode4Teaching}, sino que supondrá un punto y aparte que cierra el cuarto peldaño en la escalera evolutiva de \textit{VSCode4Teaching} y sienta la base de un nuevo trabajo, ya que será la aplicación sobre la que edificaré mi Trabajo Fin de Máster.
