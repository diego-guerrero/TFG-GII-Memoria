\section{Verificación de \textit{software} y funcionalidad}
\label{sec:verificacion}
Muchos teóricos de la ingeniería del \textit{software} afirman que la verificación del código mediante pruebas es indispensable para poder garantizar su calidad. Por ejemplo, Robert C. Martin afirma que ``las pruebas unitarias son necesarias para garantizar que el código es flexible, mantenible y reutilizable, ya que eliminan el miedo a hacer modificaciones en el código'' \cite{CleanCode}.

La nueva aplicación web implementada durante el presente hito evolutivo de \textit{VSCode4Teaching} no incluye pruebas automáticas, ya que se ha primado alcanzar una versión funcional de la aplicación por encima de la verificación de su comportamiento, que requerirá de la implementación de pruebas de tres tipos: unitarias, para corroborar el funcionamiento de la lógica de negocio; de integración, para la comprobación de la correcta comunicación con el servidor; y \textit{end-to-end} o de interfaz, con el fin de verificar que la interfaz de usuario se adecía a los requerimientos específicos de cada situación; pudiendo verse complementadas por otros tipos de pruebas, como las de carga o rendimiento, las de accesibilidad o las pruebas de funcionamiento en múltiples navegadores, especialmente importantes en este caso por las divergencias de compatibilidad de las tecnologías empleadas. La incorporación de pruebas automáticas para la garantía de la calidad de la aplicación web es uno de los primeros trabajos a futuro del proyecto, tal como recoge la \referenciaSeccion{subsec:trabajosFuturos}.

El servidor y la extensión cuentan con pruebas automáticas implementadas. Estas pruebas se basan en el uso de aserciones, que son comparaciones entre los valores obtenidos al ejecutar las pruebas y los valores deseados; y en dobles, que son piezas \textit{software} que permiten sustituir las dependencias empleadas que queden fuera del ámbito de la prueba realizada para simular su comportamiento en un escenario real y poder ejecutar la prueba al completo proporcionando valores conocidos a la salida de la dependencia reemplazada.

Las pruebas implementadas son de dos tipos: unitarias, que son aquellas que tienen como alcance una sola capa del \textit{software} y que emplean dobles para simular el funcionamiento de sus dependencias; y de integración, que verifican cómo interactúan entre sí las distintas capas del mismo componente o cómo se produce la comunicación entre el componente verificado y otros agentes \textit{software}.

\subsection{Pruebas automáticas del servidor}
\label{subsec:testingServidor}
Tal como consta en la \referenciaSeccion{subsec:tecServidor}, el servidor hace uso de JUnit para la implementación y ejecución de las pruebas automáticas implementadas en el servidor. Esta biblioteca incorpora una amplia cantidad de aserciones y permite generar dobles de dependencias de forma sencilla para el implementador. El servidor incluye \textit{tests} de tres tipos:
\begin{itemize}
    \item Pruebas sobre controladores. Incluidas en el paquete \texttt{controllertests}, son pruebas unitarias que verifican el correcto funcionamiento de la capa de los controladores REST de Spring y que se implementan ejecutando llamadas HTTP a la aplicación y utilizando dobles que suplantan el funcionamiento de los servicios que emplean para la generación de la respuesta.
    \item Pruebas sobre servicios. Localizadas en el paquete \texttt{servicetests}, son pruebas unitarias que buscan verificar el correcto funcionamiento de la capa de los servicios Spring, que es la que incluye la traslación al \textit{software} de la lógica de negocio, y se implementan mediante la suplantación con dobles de los DAO de la aplicación, proporcionando instancias basadas en un conjunto de valores conocidos.
    \item Pruebas de integración. Ubicadas en el paquete \texttt{integrationtests}, y al contrario que las anteriores, son pruebas que verifican el funcionamiento de la aplicación en su integridad, lanzando una petición HTTP y sin proporcionar ningún tipo de doble. Emplean un mecanismo para la inicialización de valores conocidos en una base de datos embebida en la aplicación e instanciada únicamente durante el lanzamiento de estas pruebas, hecho que posibilita la ejecución de pruebas que validan la correcta interacción entre las capas de la arquitectura y, además, con el sistema de persistencia.
\end{itemize}

Cuantitativamente, la batería de pruebas automáticas del servidor contiene un total de 96 pruebas que alcanzan un $78,3\%$ de las clases y un $79,4\%$ de los métodos que conforman este componente.

\subsection{Pruebas automáticas de la extensión}
\label{subsec:testingExtension}
Análogamente al caso anterior, y tal como enuncia la \referenciaSeccion{subsec:tecExtension}, la extensión para Visual Studio Code también incluye una batería de pruebas automáticas implementada sobre Jest que permite verificar el correcto funcionamiento de este componente en su integridad.

Las pruebas incorporadas a la extensión permiten verificar el correcto funcionamiento de su arquitectura en su práctica totalidad, para lo que incorpora 129 pruebas automáticas. Entre ellas, cabe reseñar una cobertura del código superior al $80\%$ sobre el cliente empleado para el intercambio de peticiones con el servidor, sobre algunos de los elementos empleados en la interfaz de usuario (como los elementos mostrados en la barra de actividad) y sobre el modelo del dominio. El dato de cobertura total de la extensión es de un $61,7\%$ de las líneas de código del proyecto y de un $54,9\%$ de las funciones que incorpora.
