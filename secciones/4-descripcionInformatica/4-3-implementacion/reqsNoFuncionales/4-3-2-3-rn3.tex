\subsubsection{\texttt{RN-3}: encriptación de \textit{tokens} JWT para autenticación}
\label{subsec:rn3}

El servidor de \textit{VSCode4Teaching} es \textit{stateless}; es decir, no almacena información sobre las interacciones realizadas por los usuarios con objeto de condicionar las siguientes, sino que cada una de las peticiones realizadas debe ser autocontenida e incluye en sí misma mediante sus cabeceras y cuerpo (si lo tiene) todo aquello que se debe tener en cuenta para proporcionar una respuesta, de modo que ninguna respuesta dependerá de las peticiones anteriormente realizadas \cite{Stateless}.

Si \textit{VSCode4Teaching} verifica esta característica es porque emplea \textit{tokens} JWT (siglas de \textit{JSON web token}, del inglés ``\textit{token} web en JSON'') para la autenticación de los usuarios. Estos \textit{tokens} son piezas de información representadas como cadenas de caracteres que permiten autenticar a los usuarios de forma autocontenida, ya que incluyen en sí mismos un \textit{payload} o carga útil que recoge qué usuario es el que está realizando una determinada petición. Se generan durante el inicio de sesión y quedan firmados mediante una clave privada, por lo que la suplantación de un usuario mediante un \textit{token} distinto del expedido durante la autenticación es inviable.

Sobre este mecanismo de autenticación preexistente, el presente requisito implementa el soporte a \textit{tokens} que, una vez generados, son encriptados nuevamente mediante un cifrado simétrico para añadir una capa adicional de seguridad. Este nuevo cifrado permite obtener cadenas no directamente comprensibles como \textit{tokens} JWT, añadiendo al servidor la lógica necesaria para manejar la presencia de cabeceras personalizadas llamadas \texttt{Encrypted-Authorization} en las peticiones procedentes de la aplicación web mediante la API REST y, además, como parámetro en la llamada inicial de las conexiones mediante \textit{Web Socket}.
