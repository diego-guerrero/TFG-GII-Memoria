\renewcommand{\bibname}{Bibliografía}
\phantomsection
\begin{thebibliography}{99}
    \addcontentsline{toc}{chapter}{Bibliografía}
    % 1. Introducción
    \bibitem{ONUDUDH} Asamblea General de la Organización de las Naciones Unidas. (Diciembre de 1948). \textit{Declaración Universal de los Derechos Humanos} (217 [III] A). París, Francia. \href{https://www.un.org/es/about-us/universal-declaration-of-human-rights}{https://un.org}.
    \bibitem{ONU2030} Asamblea General de la Organización de las Naciones Unidas. (Octubre de 2015). \textit{Transformar nuestro mundo: la Agenda 2030 para el Desarrollo Sostenible} (A/RES/70/1). Nueva York, Estados Unidos de América. \href{https://undocs.org/es/A/RES/70/1}{https://undocs.org}.
    \bibitem{UITConectividad} Unión Internacional de Telecomunicaciones (ITU). (2022). \textit{Global Connectivity Report 2022}. \href{https://www.itu.int/hub/publication/d-ind-global-01-2022/}{https://itu.int}.
    \bibitem{LOMLOE} Ley Orgánica 3/2020, por la que se modifica la Ley Orgánica 2/2006, de 3 de mayo, de educación. (29 de diciembre de 2020, publicado en BOE núm. 340, de 30 de diciembre de 2020, páginas 122868 a 122953). \href{https://www.boe.es/eli/es/lo/2020/12/29/3/}{https://boe.es}.
    \bibitem{Prensky} Prensky, M. (2001). \textit{Digital Natives, Digital Immigrants}. \href{https://www.marcprensky.com/writing/Prensky%20-%20Digital%20Natives,%20Digital%20Immigrants%20-%20Part1.pdf}{https://marcprensky.com}.
    \bibitem{CAMCurriculoESO} Decreto 65/2022, por el que se establecen para la Comunidad de Madrid la ordenación y el currículo de la Educación Secundaria Obligatoria. (20 de julio de 2022, publicado en BOCM núm. 176, de 26 de julio de 2022, páginas 396 a 716) \href{https://www.bocm.es/boletin/CM_Orden_BOCM/2022/07/26/BOCM-20220726-2.PDF}{https://bocm.es}.
    \bibitem{CompDigEduINTEF} Instituto Nacional de Tecnologías Educativas y de Formación del Profesorado. (2024). \textit{¿Qué es \#CompDigEdu? - INTEF}. \href{https://intef.es/competencia-digital-educativa/compdigedu/}{https://intef.es/}.
    \bibitem{TFG_Ivan} Chicano Capelo, I. (2020). \textit{VS Code 4 Teaching: Los ejercicios directos al editor}.
    \bibitem{TFG_Alvaro} Rivas Alcobendas, Á. J. (2021). \textit{VSCode4Teaching 2.0: Seguimiento de ejercicios de alumnos en el IDE del profesor}.
    \bibitem{TFG_Diego1} Guerrero Carrasco, D. (2024). \textit{VSCode4Teaching: mantenimiento y evolución de la herramienta para la enseñanza de la programación en línea}.
    \bibitem{JetBrains_DevState} JetBrains, Inc. (2023). \textit{Developer State Survey 2023}. \href{https://www.jetbrains.com/lp/devecosystem-2023/}{https://jetbrains.com}.
    \bibitem{JetBrains_Plugins} JetBrains, Inc. (2024). \textit{Developing a Plugin - IntelliJ Platform Plugin SDK}. \href{https://plugins.jetbrains.com/docs/intellij/developing-plugins.html}{https://jetbrains.com}.

    % 3.1. Tecnologías
    % 3.1.1. Aplicación web Angular
    % 3.1.1.1. Angular
    \bibitem{Tec_Angular} Google, Inc. (2024). \textit{What is Angular? - Angular}. \href{https://angular.dev/overview}{https://angular.dev}.
    \bibitem{SPA_Ventajas} Davidson, T. (28 de febrero, 2023). Single Page Application (SPA) vs Multi Page Application (MPA): Which is the best? \textit{CleanCommit}. \href{https://cleancommit.io/blog/spa-vs-mpa-which-is-the-king}{https://cleancommit.io}.
    \bibitem{Angular_Componentes} Google, Inc (2024). \textit{Composing with Components - Angular}. \href{https://angular.dev/essentials/components}{https://angular.dev}.
    \bibitem{WebComponents} Mozilla Developer Network (2024). \textit{Web Components}. \href{https://developer.mozilla.org/en-US/docs/Web/API/Web_components}{https://mozilla.org}.
    \bibitem{subsec:tecAppAngularSurvey} Stack Overflow. (2023). \textit{Stack Overflow Developer Survey 2023}. \href{https://survey.subsec:tecAppAngular.co/2023}{https://subsec:tecAppAngular.co}.
    \bibitem{AngularReact} Reis, J; Figueiredo, R. (15 de abril, 2024). Angular vs React: a comparison of both frameworks. \textit{ImaginaryCloud}. \href{https://www.imaginarycloud.com/blog/angular-vs-react/}{https://imaginarycloud.com}.
    \bibitem{React} Meta Open Source (2024). \textit{React}. \href{https://react.dev}{https://react.dev}.
    \bibitem{SemVer} Preston-Werner, T. (2023). \textit{Semantic Versioning 2.0.0}. \href{https://semver.org}{https://semver.org}.
    % 3.1.1.2. FSA API
    \bibitem{UsoNavegadores} StatCounter. (2023). \textit{Desktop browser market share worldwide}. \href{https://gs.statcounter.com/browser-market-share/desktop/worldwide/2023}{https://statcounter.com}.
    \bibitem{WebKit} Apple, Inc. (2024). \textit{WebKit}. \href{https://webkit.org}{https://webkit.org}.
    \bibitem{Gecko} Mozilla Foundation. (2024). \textit{Gecko - Firefox Source Docs documentation}. \href{https://firefox-source-docs.mozilla.org/overview/gecko.html}{https://mozilla.org}.
    \bibitem{Chromium} Google, Inc. (2024). \textit{Chromium}. \href{https://www.chromium.org/Home/}{https://chromium.org}.
    \bibitem{Blink} Google, Inc. (2024). \textit{Blink (Rendering Engine)}. \href{https://www.chromium.org/blink/}{https://chromium.org}.
    \bibitem{EdgeChromium} Rajaa, R. (17 de enero, 2020). Microsoft Edge 79: What's new in the Chromium-based Edge? \textit{BrowserStack}. \href{https://www.chromium.org/blink/}{https://chromium.org}.
    \bibitem{SpiderMonkey} Mozilla Foundation. (2024). \textit{SpiderMonkey - Firefox Source Docs documentation}. \href{https://firefox-source-docs.mozilla.org/js/index.html}{https://mozilla.org}.
    \bibitem{V8} Google, Inc. (2024). \textit{V8 JavaScript engine}. \href{https://v8.dev}{https://v8.dev}.
    \bibitem{FileSystemAPI} Web Hypertext Application Technology Working Group. (24 de enero, 2024). \textit{File System API living standard}. \href{https://fs.spec.whatwg.org}{https://whatwg.org}.
    \bibitem{WICG} Web, Incubator Community Group. (2024). \textit{Web, Incubator Community Group}. \href{https://wicg.io}{https://wicg.io}.
    \bibitem{FileSystemAccessAPI} Web, Incubator Community Group. (20 de marzo, 2024). \textit{File System Access} (draft community group report). \href{https://wicg.github.io/file-system-access/}{https://wicg.github.io}.
    \bibitem{ArticuloChromeFsaAPI} LePage, P., Steiner, T. (2020). \textit{The File System Access API: simplifying access to local files}. \href{https://developer.chrome.com/docs/capabilities/web-apis/file-system-access}{https://developer.chrome.com}.
    \bibitem{ChromeLabsEditor} Lepage, P., Kruisselbrink, M. (2020). \textit{GoogleChromeLabs/text-editor} - \textit{HTML5 Text Editor}. (Repositorio de código en GitHub). \href{https://github.com/GoogleChromeLabs/text-editor}{https://github.com}.
    \bibitem{Polyfill} Sharp, R. (8 de octubre, 2010). \textit{What is a Polyfill?} \href{https://remysharp.com/2010/10/08/what-is-a-polyfill}{https://remysharp.com}.
    \bibitem{Ponyfill} Sorhus, S. (29 de septiembre, 2016). \textit{sindresorhus/ponyfill} - \textit{Ponyfill}. (Repositorio de código en GitHub). \href{http://ponyfill.com}{http://ponyfill.com}.
    \bibitem{BrowserFSAccess} Steiner, T. et al. (2020). \textit{GoogleChromeLabs/browser-fs-access} - \textit{Browser-FS-Access: File System Access API with legacy fallback in the browser}. (Repositorio de código en GitHub). \href{https://github.com/GoogleChromeLabs/browser-fs-access}{https://github.com}.
    % 3.1.1.3. RxJS
    \bibitem{ReactiveProgramming} Otta, M., Martin, E. (23 de mayo, 2024). \textit{What is Reactive Programming?} \textit{Baeldung}. \href{https://www.baeldung.com/cs/reactive-programming}{https://baeldung.com}.
    \bibitem{GammaObserver} Gamma, E., Helm. R et al. (1995). \textit{Observer}. En \textit{Design Patterns} (293-303). Addison-Wesley.
    \bibitem{RxJS} ReactiveX. (2024). \textit{RxJS - Introduction}. \href{https://rxjs.dev/guide/overview}{https://rxjs.dev}.
    \bibitem{ReactiveX} ReactiveX. (2024). \textit{ReactiveX - Intro}. \href{https://reactivex.io/intro.html}{https://reactivex.io}.
    \bibitem{Promise} Mozilla Foundation. (8 de agosto, 2023). \textit{Promise - JavaScript}. \href{https://developer.mozilla.org/docs/Web/JavaScript/Reference/Global_Objects/Promise}{https://mozilla.org}.
    \bibitem{ES6} ECMA International. (Junio, 2015). \textit{ECMA-262 Standard: ECMAScript 2015 Language Specification}. \href{https://ecma-international.org/wp-content/uploads/ECMA-262_6th_edition_june_2015.pdf}{https://ecma-international.org}.
    % 3.1.1.4. TS
    \bibitem{TypeScript} Microsoft, Inc. (2024). \textit{TypeScript: JavaScript with Syntax for Types}. \href{https://www.typescriptlang.org}{https://typescriptlang.org}.
    % 3.1.1.5. Node
    \bibitem{Node} OpenJS Foundation. (2024). \textit{Node.js - Run JavaScript Everywhere}. \href{https://nodejs.org/en}{https://nodejs.org}.
    \bibitem{TIOBE} Tiobe Software BV. (1 de junio, 2024). \textit{TIOBE Index - June 2024 - TIOBE}. \href{https://www.tiobe.com/tiobe-index}{https://tiobe.com}.
    \bibitem{npmjscom} npm, Inc. (2024). \textit{npm About}. \href{https://www.npmjs.com/about}{https://npmjs.com}.
    % 3.1.1.6. SCSS, BS y Chart.js
    \bibitem{CSS_W3C} World Wide Web Consortium (2024). \textit{Cascading Style Sheets - Specs}. \href{https://www.w3.org/Style/CSS/#specs}{https://w3.org}.
    \bibitem{Sass} The Sass Team (2024). \textit{Syntactically Awesome Style Sheets - Syntax}. \href{https://sass-lang.com/documentation/syntax}{https://sass-lang.com}.
    \bibitem{Bootstrap} The Bootstrap Team (2024). \textit{Bootstrap}. \href{https://getbootstrap.com}{https://getbootstrap.com}.
    \bibitem{ChartJS} Downie, N. (2013). \textit{Chart.js}. \href{https://www.chartjs.org}{https://www.chartjs.org}.
    % 3.1.2.1. MySQL
    \bibitem{MySQL} Oracle Corporation. (2024). \textit{MySQL Community Edition}. \href{https://www.mysql.com/products/community/}{https://mysql.com}.
    \bibitem{PostgreSQL} PostgreSQL Global Development Group. (2024). \textit{PostgreSQL: About}. \href{https://www.postgresql.org/about/}{https://postgresql.org}.
    % 3.1.2.2. Spring-Java
    \bibitem{Java} Oracle Corporation. (2024). \textit{What is Java and why do I need it?} \href{https://www.java.com/download/help/whatis_java.html}{https://java.com}.
    \bibitem{SpringFramework} Broadcom, Inc. (2024). \textit{Spring Framework}. \href{https://spring.io/projects/spring-framework}{https://spring.io}.
    \bibitem{SpringBoot} Broadcom, Inc. (2024). \textit{Spring Boot}. \href{https://spring.io/projects/spring-boot}{https://spring.io}.
    \bibitem{SpringData} Broadcom, Inc. (2024). \textit{Spring Data}. \href{https://spring.io/projects/spring-data}{https://spring.io}.
    % 3.1.2.3. Maven
    \bibitem{Maven} Apache Software Foundation. (2024). \textit{Maven - Welcome to Apache Maven}. \href{https://maven.apache.org}{https://apache.org}.
    % 3.1.2.4. JUnit
    \bibitem{JUnit} JUnit Team. (2024). \textit{JUnit 5 User Guide}. \href{https://junit.org/junit5/docs/current/user-guide/}{https://junit.org}.
    % 3.1.3. Extensión
    \bibitem{Jest} Open JS Foundation. (2024). \textit{Jest · Delightful JavaScript Testing}. \href{https://jestjs.io/}{https://jestjs.io}.
    
    % 3.1.4. Divulgación, despliegue y distribución
    \bibitem{GitHub} GitHub, Inc. (2024). \textit{About GitHub and Git - GitHub Docs}. \href{https://docs.github.com/en/get-started/start-your-journey/about-github-and-git}{https://github.com}
    \bibitem{GitHubCifras} GitHub, Inc. (2024). \textit{About GitHub}. \href{https://github.com/about}{https://github.com/about}
    \bibitem{Docker} Docker, Inc. (2024). \textit{Docker: Accelerated, Containerized Application Development}. \href{https://www.docker.com}{https://docker.com}
    \bibitem{DockerEngine} Docker, Inc. (2024). \textit{Docker Engine overview}. \href{https://docs.docker.com/engine/}{https://docker.com}
    \bibitem{DockerContainers} Docker, Inc. (2024). \textit{What is a Container?} \href{https://www.docker.com/resources/what-container/}{https://docker.com}
    \bibitem{Dockerfile} Docker, Inc. (2024). \textit{Packaging your software: Dockerfile} \href{https://docs.docker.com/build/building/packaging/#dockerfile}{https://docker.com}
    \bibitem{DockerCompose} Docker, Inc. (2024). \textit{Docker Compose overview}. \href{https://docs.docker.com/compose/}{https://docker.com}.
    \bibitem{DockerHub} Docker, Inc. (2024). \textit{Docker Hub overview}. \href{https://docs.docker.com/docker-hub/}{https://docker.com}.
    \bibitem{VSCodeMarketplace} Microsoft, Inc. (2024). \textit{Managing extensions in Visual Studio Code: Visual Studio Code Marketplace}. \href{https://code.visualstudio.com/docs/editor/extension-marketplace}{https://visualstudio.com}.

    % 3.2. Herramientas
    \bibitem{WebStorm} JetBrains, Inc. (2024). \textit{WebStorm: the JavaScript and TypeScript IDE}. \href{https://www.jetbrains.com/webstorm/}{https://jetbrains.com}.
    \bibitem{VSCode} Microsoft, Inc. (2024). \textit{Visual Studio Code: Code editing. Redefined}. \href{https://code.visualstudio.com/}{https://visualstudio.com}.
    \bibitem{IntelliJ} JetBrains, Inc. (2024). \textit{IntelliJ IDEA: the Leading Java and Kotlin IDE}. \href{https://www.jetbrains.com/idea/}{https://jetbrains.com}.
    \bibitem{Git} Software Freedom Conservancy. (2024). \textit{Git}. \href{https://git-scm.com/about/free-and-open-source/}{https://git-scm.com}.
    \bibitem{DevOps} Atlassian. (2024). \textit{What is DevOps}. \href{https://www.atlassian.com/devops}{https://atlassian.com}.
    \bibitem{CICDRedHat} Red Hat, Inc. (December 12, 2023). \textit{What is CI/CD?} \href{https://www.redhat.com/en/topics/devops/what-is-ci-cd}{https://redhat.com}.
    \bibitem{CICDGitlab} GitLab B.V. (2024). \textit{What is CI/CD?} \href{https://about.gitlab.com/topics/ci-cd/}{https://gitlab.com}.
    \bibitem{GitHubActions} GitHub, Inc. (2024). \textit{GitHub Actions}. \href{https://github.com/features/actions/}{https://github.com}.
    \bibitem{Trello} Atlassian. (2024). \textit{What is Trello: learn features, uses \& more}. \href{https://trello.com/en/tour}{https://trello.com}.

    % 3.3. Metodología
    \bibitem{ModeloEspiral} Boehm, B. (1 de agosto, 1986). ``\textit{A spiral model of software development and enhancement}''. \textit{ACM SIGSOFT Software Engineering Notes}, vol. 11 (nº 2), pp. 22-42. \href{https://dl.acm.org/doi/10.1145/12944.12948}{https://acm.org}.
    \bibitem{RUP} Booch, G., Jacobson, I., Rumbaugh, J. (1999). \textit{The Unified Software Development Process}.
    \bibitem{UML} Booch, G., Jacobson, I., Rumbaugh, J. (2005). \textit{The Unified Modeling Language User Guide (2nd edition)}.
    \bibitem{XP} Beck, K. (1999). \textit{Extreme Programming explained: embrace change}.
    \bibitem{AgileManifesto} Beck, K., et al. (2001). \textit{Manifesto for Agile Software Development}. \href{https://agilemanifesto.org}{https://agilemanifesto.org}.

    % 4. Descripción informática
    % 4.1.1. Requisitos funcionales
    \bibitem{XP_HistoriasUsuario} Sergeev, A. (26 de mayo de 2016). \textit{Extreme programming user stories}. \href{https://hygger.io/blog/extreme-programming-user-stories/}{https://hygger.io}.
    % 4.1.2. Requisitos no funcionales
    \bibitem{DockerfileMultistage} Docker, Inc. (2024). \textit{Multi-stage builds} \href{https://docs.docker.com/build/building/multi-stage}{https://docker.com}.
    % 4.2. Arquitectura
    \bibitem{MVC} Fowler, M. (2002). \textit{Patterns of enterprise application architecture}. Addison-Wesley Professional. \href{https://www.oreilly.com/library/view/patterns-of-enterprise/0321127420/}{https://oreilly.com}.
    % 4.3.2.3. RN-3
    \bibitem{Stateless} Red Hat, Inc. (21 de diciembre, 2023). \textit{Stateful vs stateless}. \href{https://www.redhat.com/en/topics/cloud-native-apps/stateful-vs-stateless}{https://redhat.com}.
    % 4.4. Verificación
    \bibitem{CleanCode} Martin, R. C. (2009). \textit{Unit Tests}. En \textit{Clean Code} (121-134). Prentice Hall.
    % 4.5. Distribución y despliegue
    \bibitem{ApacheLicense} Apache Software Foundation. (2004). \textit{Apache License 2.0}. \href{https://www.apache.org/licenses/LICENSE-2.0.txt}{https://apache.org}.
    \bibitem{FreeSoftwareFreedoms} Free Software Foundation, Inc. (1996). \textit{¿Qué es el software libre?} \href{https://www.gnu.org/philosophy/free-sw.es.html#four-freedoms}{https://gnu.org}.

    % 5. Conclusiones y trabajos futuros
    \bibitem{MonacoEditor} Microsoft, Inc. (2024). \textit{Monaco Editor}. \href{https://microsoft.github.io/monaco-editor/}{https://microsoft.github.io}.
    \bibitem{Moodle} Moodle. \textit{Moodle - Open-source learning platform}. \href{https://moodle.org}{https://moodle.org}.
    \bibitem{LTI_Spec} 1EdTech. (16 de abril de 2019). \textit{Learning Tools Interoperability Core Specification} (versión 1.3). \href{https://www.imsglobal.org/spec/lti/v1p3}{https://imsglobal.org}.
\end{thebibliography}

\raggedbottom
% \afterpage{\blankpage}
\newpage
